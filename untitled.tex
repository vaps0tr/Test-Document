\section{Heading First}
Hey, welcome. Hi Katie. 


hi

hello TIM! 

Weird Not sure how we can both edit

\textbf{Does everyone on the team know LaTeX?}


Double click anywhere on the text to start writing. In addition to simple text you can also add text formatted in \textbf{boldface}, \textit{italic}, and yes, math too: $E  =  mc^{2}$! Add images by drag'n'drop or click on the "Insert Figure" button. 

This is a test paragragh. I want to add a figure: Figure \ref{253882}.

How did you add that reference to figure 1? I can't figure out how to add a \verb{\label{}} to the figure, as I did to the table, and those numbers are totally unwieldy... -ELG

I suggest that for labeling in general we use the following:
\begin{itemize}
\item For tables, use "tab:table_name"
\item For figures, use "fig:figure_name"
\item For equations, use "eqn:equation_name"
\item etc.
\end{itemize}

Testing citations which is very important. \cite{Eisenhardt1989} So you have to use the Mendeley command "Copy As..." and then BibTex record. Then this will work. -ELG

I am happy to use this, but I think we need a little more testing particularly of integration with Word. One big concern is I'm not clear whether it's at all portable offline. With sharelatex, you're editing a normal LaTeX document, so you can download and use in a LaTeX editor. With this, it's not a complete LaTeX doc. The other concern is related. Do you have the same kind of control over the document template, the citation formats, and everything else that you have with LaTeX? Or maybe we don't care about that because we plan to put it into Word anyway? I'm just worried because what if this tool suddenly goes away. The others have more stable options. But I'm not arguing against it, just raising concerns that hopefully someone can investigate/mitigate. -ELG

The export options are Word, PDF, LaTex, and "Zip". It seems like you could download this and edit in a normal LaTex editor. The backend is a LaTex document =
